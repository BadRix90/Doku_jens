% !TEX root = ../Projektdokumentation.tex
\section{Entwurfsphase} 
\label{sec:Entwurfsphase}

\subsection{Zielplattform}
\label{sec:Zielplattform} 
Die Zielplattform für dieses Projekt ist die IT-Infrastruktur der VONMÄHLEN GmbH, die auf ein internes Netzwerk 
ausgerichtet ist. Kernkomponenten des Projekts, darunter die PostgreSQL-Datenbank und das \ac{ERP}-System BC, sind in 
die bestehende Serverarchitektur integriert. Zur Automatisierung von Aufgaben wie dem täglichen Datenabgleich und 
der Aktualisierung der Trackinginformationen werden Docker und Apache Airflow eingesetzt. Diese Werkzeuge ermöglichen 
eine nahtlose Integration und die zuverlässige Ausführung wiederkehrender Aufgaben. Der Zugriff auf Business Central 
erfolgt über eine abgesicherte \ac{API}-Schnittstelle, während der Datentransfer über einen SFTP-Server abgewickelt wird, 
der die Trackingdaten vom externen Dienstleister sicher aufnimmt. Mit dieser Infrastruktur wird eine optimale Sicherheit 
und Performance erreicht, und gleichzeitig können die Anforderungen an eine hohe Verfügbarkeit und Skalierbarkeit erfüllt 
werden.

\subsection{Architekturdesign}
\label{sec:Architekturdesign}
Python in Verbindung mit Apache Airflow wurde aufgrund der einfachen Integration und Technologiengleichheit im Unternehmen gewählt. 
Python ermöglicht schnelle Entwicklung, während Airflow die Orchestrierung und Überwachung der Datenpipelines übernimmt. Ideal für 
die automatisierte Verarbeitung der Tracking-Daten.

Zusätzlich wurde Docker als Plattform gewählt, um eine isolierte, portable und skalierbare Umgebung für die Anwendungen bereitzustellen. 
Docker erleichtert die Verwaltung der eingesetzten Dienste wie PostgreSQL und Apache Airflow und ermöglicht eine schnelle Bereitstellung 
und Wiederherstellung der Anwendung. Die Containerisierung stellt sicher, dass die Anwendung unabhängig von der Umgebung läuft und 
problemlos erweitert werden kann.

PostgreSQL wurde als Datenbankmanagementsystem ausgewählt, da es eine leistungsstarke, zuverlässige und kostenlose Open-Source-Lösung darstellt. 
Es bietet umfangreiche Funktionen wie Transaktionssicherheit, Skalierbarkeit und eine hohe Leistung bei der Verarbeitung großer Datenmengen. 
Die Fähigkeit von PostgreSQL, komplexe Abfragen effizient zu verarbeiten, war entscheidend für die Auswahl, da die Tracking-Daten sowohl 
regelmäßig aktualisiert als auch konsolidiert werden müssen. Zusammen mit Docker ergibt sich eine robuste, skalierbare und leicht wartbare 
Architektur, die den Anforderungen des Projekts optimal entspricht.