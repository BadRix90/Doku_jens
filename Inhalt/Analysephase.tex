% !TEX root = ../Projektdokumentation.tex
\section{Analysephase} 
\label{sec:Analysephase}

\subsection{Ist-Analyse} 
\label{sec:IstAnalyse}
Derzeit werden täglich alle Warensendungen manuell geprüft, indem Lieferungen mit Trackingnummern 
herausgefiltert und im Onlineportal des externen Versanddienstleisters überprüft werden. 
Mitarbeiter in der Abteilung \ac{SCM} durchforsten Verkaufsaufträge nach 
Trackingnummern und prüfen im Versandportal den aktuellen Status der Sendungen. Der Fokus liegt 
hierbei darauf, Unregelmäßigkeiten wie Verzögerungen oder Fehlzustellungen zu identifizieren. 
Bei auffälligen Zuständen wird die zuständige Abteilung informiert, um die erforderlichen Maßnahmen einzuleiten.

\subsection{Wirtschaftlichkeitsanalyse}
\label{sec:Wirtschaftlichkeitsanalyse}
Die Wirtschaftlichkeit des Projekts ließ sich unkompliziert anhand der eingesparten Arbeitszeit der SCM-Mitarbeiter berechnen, 
da diese durch die Automatisierung des Prüfungsprozesses deutlich reduziert wird.


\subsection{Kostenanalyse}
\label{sec:Kostenanalyse}
Die Kosten für die Implementierung des Systems umfassen zwei Hauptkategorien: 
Initiale Implementierungskosten und laufende Betriebskosten.

\subsubsection{Einmalige Implementierungskosten}
\label{sec:Implementierungskosten}

Diese Kosten fallen nur einmal an und umfassen die Entwicklung des Systems, 
die Anschaffung notwendiger Software sowie die Einrichtung der Infrastruktur.

	\textbf{Entwicklungskosten (inklusive Planung, Design und Implementierung)}:
    \begin{itemize}
        \item Auszubildender: 80 Stunden a 7 €/Stunde = 560 €
        \item Entwickler-Support: 10 Stunden a 22 €/Stunde = 220 €
        \item Fachabteilung SCM: 2 Stunden a 30 €/Stunde
    \end{itemize}
    \clearpage

	\textbf{Softwarelizenzen und Tools:}
    \begin{itemize}
        \item PostgreSQL: Kostenlos
        \item Visual Studio Code: Kostenlos
        \item Docker: Kostenlos
        \item Apache Airflow: Kostenlos
        \item Business Central Lizenz (während der Implementierung): 65,50 €/Monat, über 2 Monate = 131 € 
    \end{itemize}

	\textbf{Hardware- und Infrastrukturkosten:}
    \begin{itemize}
        \item Serverkosten, Server vorhanden: 0€/Jahr
    \end{itemize}
        
	\textbf{Gesamtkosten für die Implementierung: 971 €}

\subsubsection{Laufende Betriebskosten}
\label{Betriebskosten}

Diese Kosten fallen während des Betriebs des Systems an.

\textbf{Wartung, Fehlerbehebung und Optimierungen:}
\begin{itemize}
    \item Entwickler-Support (12 Stunden pro Jahr a 22€/Stunde) = 264 €/Jahr
    \item Erweiterung des bestehenden Hetzner Servers 120 €/Jahr
\end{itemize}

\textbf{Gesamtkosten für den laufenden Betrieb: 384 €/Jahr }

\clearpage

\subsection{Vergleich: Manuelle vs. Automatisierte Prozesse}
\label{vergleichProzesse}
Tabelle~\ref{tab:vergleichProzesse}
\tabelle{Vergleich von manuellen und automatisierten Prozessen}{tab:vergleichProzesse}{vergleichProzesse}\\



















