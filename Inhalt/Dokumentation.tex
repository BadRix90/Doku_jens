% !TEX root = ../Projektdokumentation.tex
\section{Dokumentation}
\label{sec:Dokumentation}


\subsection{Zielgruppe der Entwicklerdokumentation}
\label{sec:ZielgruppeDoku}

Die Dokumentation richtet sich an administratives Personal der VONMÄHLEN GmbH. 
Der Umgang mit Apache Airflow und PostgreSQL ist diesem Personenkreis geläufig. 
Python wird als unternehmensübliche Programmiersprache genutzt. Die Dokumentation ist 
sowohl im Markdown-Format im GIT-Repository (\textit{README.md}) als auch im internen Unternehmens-Wiki verfügbar.

\subsection{Inhalt der Entwicklerdokumentation}
\label{sec:Entwicklerdokumentation}

Die Entwicklerdokumentation umfasst:
\begin{itemize}
    \item Integration in Apache Airflow: Anleitung zur Einbindung der Software in die bestehende Airflow-Instanz.
    \item Funktionsweise der Software: Detaillierte Beschreibung der Anwendung und ihrer Hauptfunktionen.
    \item Fehlermeldungen und Maßnahmen: Übersicht der möglichen Fehlermeldungen und empfohlene Vorgehensweisen zur Fehlerbehebung.
    \item Erweiterungsmöglichkeiten: Anleitung zur Anpassung der Software bei Änderungen der API des Versanddienstleisters, 
    wie z.B. neue Status-Codes oder geänderte CSV-Schemata.
\end{itemize}


\subsection{Code Dokumentation}
\label{sec:CodeDokumentation}

Der Programmcode enthält umfassende Python-Docstrings, die die Funktionen und Klassen erläutern. 
Diese richten sich an Personen, die den erstellten Programmcode anpassen oder besser verstehen möchten. 
Ergänzend dazu finden sich im Anhang:
\begin{itemize}
    \item A9: Entwicklerdokumentation (Auszug Firmen Wiki)
    \item A10: Entwicklerdokumentation (Auszug Git)
    \item A6: Programmcode (Auszug \textit{bc\_db.py})
    \item A7: Programmcode (Auszug \textit{trackings.py})
\end{itemize}

\clearpage