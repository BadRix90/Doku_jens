% !TEX root = ../Projektdokumentation.tex
\section{Abnahmephase}
\label{sec:Abnahmephase}

\subsection{Testphase}
\label{sec:Testphase}
Alle Systeme wurden unter realistischen Bedingungen in einer speziell dafür eingerichteten Entwicklungsumgebung getestet.
Diese Umgebung, die regelmäßig für Integration-Tests genutzt wird, stellt sicher, dass die Systeme stabil und
zuverlässig funktionieren.

Zum Testsystem gehören unter anderem:

\begin{itemize}
	\item Microsoft Dynamics 365 Business Central
	\item PostgreSQL-Datenbank (siehe Anhang)
	\item Apache Airflow für automatisierte Python-Skripte (siehe Anhang)
	\item SFTP-Server für den Datentransfer
\end{itemize}

\subsubsection{Tests mit dem Skript \textit{trackings.py}}
\label{sec:Tests mit dem Skript trackings.py}

Die Tests bezogen sich hauptsächlich auf die folgenden Aspekte:

\begin{enumerate}
    \item Upload von Dummy-Daten auf den SFTP-Server
    \begin{itemize}
        \item Im Rahmen der Tests wurden regelmäßig Dummy-Daten auf den SFTP-Server geladen, um den vollständigen Prozess von der Datenerfassung bis zur Weiterverarbeitung zu simulieren.
        \item Diese Dummy-Daten bestanden aus korrekt formatierten Tracking-Datensätzen sowie absichtlich fehlerhaften Datensätzen, um die Robustheit der Validierungslogik zu prüfen.
    \end{itemize}
    \item Abholung und Validierung der Daten durch das Skript
    \begin{itemize}
        \item Das Skript \textit{trackings.py} wurde verwendet, um die hochgeladenen Dummy-Daten vom SFTP-Server abzuholen und in die PostgreSQL-Datenbank zu speichern.
        \item Eine umfassende Validierung der Daten wurde durchgeführt, um sicherzustellen, dass nur vollständige und korrekte Datensätze verarbeitet werden.
        \item Es wurden insbesondere folgende Validierungsszenarien getestet:
        \begin{itemize}
            \item Vollständige und korrekte Datensätze: Prüfung, ob die Daten korrekt in die Datenbank eingefügt werden.
            \item Fehlerhafte oder unvollständige Datensätze: Sicherstellung, dass fehlerhafte Datensätze (z.B. fehlende Pflichtfelder, falsche Datentypen) nicht in die Datenbank übernommen werden, sondern eine entsprechende Fehlermeldung generieren.
            \item Doppelte Datensätze: Dummy-Daten, die doppelt vorhanden waren, wurden hochgeladen, um sicherzustellen, dass das Skript diese erkennt und nicht erneut in die Datenbank einfügt.
        \end{itemize}
    \end{itemize}
    \item Löschen der erfolgreich gelesenen Daten auf dem SFTP-Server
    \begin{itemize}
        \item Nach der erfolgreichen Verarbeitung der Dummy-Daten wurde das automatische Löschen der verarbeiteten Dateien auf dem SFTP-Server getestet.
        \item Es wurde sichergestellt, dass nur die Daten, die erfolgreich verarbeitet wurden, gelöscht wurden, während nicht verarbeitete oder fehlerhafte Dateien weiterhin auf dem SFTP-Server verbleiben.
        \item Das Skript wurde darauf getestet, korrekt zwischen verarbeiteten und nicht verarbeiteten Dateien zu unterscheiden und die Löschvorgänge nur bei Erfolg auszuführen.
    \end{itemize}
    \item Fehlerhafte Daten gezielt einfügen
    \begin{itemize}
        \item Es wurden gezielt falsche Daten auf den SFTP-Server hochgeladen, um zu prüfen, wie die Skripte auf inkonsistente Informationen reagieren.
        \item Ziel dieser Tests war es, sicherzustellen, dass das System solche fehlerhaften Daten erkennt und entsprechend darauf reagiert, z.B. durch Protokollierung im Log oder durch Benachrichtigung der zuständigen Personen.
    \end{itemize}
    \item Verifizierung der Datenintegrität
    \begin{itemize}
        \item Die Testdaten wurden genutzt, um die Datenintegrität in der PostgreSQL-Datenbank zu überprüfen.
        \item Es wurde sichergestellt, dass nur die validierten Daten korrekt in der Datenbank gespeichert wurden und dass keine fehlerhaften Daten Einfluss auf die vorhandenen Datensätze hatten.
        \item Dies beinhaltete auch die Konsistenzprüfung der Datenbank nach der Verarbeitung.
    \end{itemize}
\end{enumerate}

\subsubsection{Tests mit dem Skript \textit{bc\_db.py}}
\label{sec:Tests mit dem Skript bc_db.py}

Die Testphase beinhaltete die folgenden Aspekte:

\begin{enumerate}
    \item Abfrage und Verarbeitung der Tracking-Daten
    \begin{itemize}
        \item Das Skript \textit{bc\_db.py} wurde getestet, um sicherzustellen, dass es in der Lage ist, die Tracking-Daten korrekt aus der PostgreSQL-Datenbank zu lesen und zu verarbeiten.
        \item Es wurden mehrere Testläufe durchgeführt, um sicherzustellen, dass das Skript die relevanten Daten korrekt abfragt und für die Weiterverarbeitung vorbereitet.
        \item Verarbeitungsszenarien: Verschiedene Szenarien, wie z.B. Datensätze mit unterschiedlichen Statuscodes, wurden geprüft, um sicherzustellen, dass das System die jeweilige Geschäftslogik korrekt anwendet.
    \end{itemize}
    \item Synchronisation mit Microsoft Business Central
    \begin{itemize}
        \item Ein wesentlicher Bestandteil des Skripts ist die Synchronisation der Tracking-Daten mit dem ERP-System Microsoft Business Central.
        \item Die Funktionalität wurde getestet, indem die API-Aufrufe mit verschiedenen Testdaten ausgeführt wurden.
        \item Es wurde insbesondere geprüft, dass die Statusinformationen korrekt in Business Central aktualisiert werden und dass die API fehlerfrei funktioniert.
        \item Hierzu wurden auch absichtlich falsche API-Daten verwendet, um sicherzustellen, dass das System korrekt auf Fehlermeldungen reagiert und diese verarbeitet.
    \end{itemize}
    \item Fehlerbehandlung und Benachrichtigung
    \begin{itemize}
        \item Das Skript wurde getestet, um sicherzustellen, dass es robust gegenüber Fehlern ist, die bei der Verarbeitung der Daten auftreten können.
        \item Es wurden Szenarien simuliert, in denen die API-Verbindung fehlschlägt, um zu prüfen, wie das Skript darauf reagiert.
        \item Zudem wurde die Benachrichtigungslogik getestet: Wenn eine Sendung länger als die definierte Anzahl an Tagen in einem bestimmten Status verweilt, wird automatisch eine Benachrichtigung an die zuständigen Mitarbeiter gesendet.
        \item Die Testläufe stellten sicher, dass diese E-Mails korrekt ausgelöst werden.
    \end{itemize}
    \item Korrekte Handhabung von Ausnahmefällen
    \begin{itemize}
        \item Es wurden Tests durchgeführt, um sicherzustellen, dass alle möglichen Ausnahmefälle korrekt behandelt werden, z.B. wenn die Datenbankverbindung vorübergehend nicht verfügbar ist oder unerwartete Datenformate auf der API-Ebene auftreten.
        \item Die Protokollierung der Fehler wurde ebenfalls geprüft, um sicherzustellen, dass alle Fehler im Log korrekt vermerkt werden, damit diese bei Bedarf nachvollzogen und behoben werden können.
    \end{itemize}
    \item End-to-End-Test der gesamten Datenpipeline
    \begin{itemize}
        \item Schließlich wurde ein End-to-End-Test der gesamten Datenpipeline durchgeführt, bei dem beide Skripte (\textit{trackings.py} und \textit{bc\_db.py}) zusammenarbeiteten.
        \item Ziel war es sicherzustellen, dass die Daten von der Erfassung über den SFTP-Server, die Validierung und Speicherung in der PostgreSQL-Datenbank bis hin zur Synchronisation mit Business Central nahtlos verarbeitet werden.
        \item Diese Tests beinhalteten auch die Simulation realer Arbeitsbedingungen, um die Performance und Zuverlässigkeit des Systems unter hoher Last zu prüfen.
    \end{itemize}
\end{enumerate}
