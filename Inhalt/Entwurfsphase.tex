% !TEX root = ../Projektdokumentation.tex
\section{Entwurfsphase} 
\label{sec:Entwurfsphase}

\subsection{Zielplattform}
\label{sec:Zielplattform} 
Die Zielplattform für dieses Projekt ist die IT-Infrastruktur der VONMÄHLEN GmbH, die auf ein internes Netzwerk 
ausgerichtet ist. Kernkomponenten des Projekts, darunter die PostgreSQL-Datenbank und das \ac{ERP}-System BC, sind in 
die bestehende Serverarchitektur integriert. Zur Automatisierung von Aufgaben wie dem täglichen Datenabgleich und 
der Aktualisierung der Trackinginformationen werden Docker und Apache Airflow eingesetzt. Diese Werkzeuge ermöglichen 
eine nahtlose Integration und die zuverlässige Ausführung wieder-kehrender Aufgaben. Der Zugriff auf Business Central 
erfolgt über eine abgesicherte \ac{API}-Schnittstelle, während der Datentransfer über einen SFTP-Server abgewickelt wird, 
der die Trackingdaten vom externen Dienstleister sicher aufnimmt. Mit dieser Infrastruktur wird eine optimale Sicherheit 
und Performance erreicht, und gleichzeitig können die Anforderungen an eine hohe Verfügbarkeit und Skalierbarkeit erfüllt 
werden. Architekturdesign Beschreibung und Begründung der gewählten Anwendungsarchitektur \ac{MVC}. Ggfs. Bewertung und 
Auswahl von verwendeten Frameworks sowie ggfs. eine kurze Einführung in die Funktionsweise des verwendeten Frameworks.

\subsection{Architekturdesign}
\label{sec:Architekturdesign}
Python in Verbindung mit Apache Airflow wurde aufgrund der einfachen Integration und Technologiengleichheit im Unternehmen gewählt. 
Python ermöglicht schnelle Entwicklung, während Airflow die Orchestrierung und Überwachung der Datenpipelines übernimmt. Ideal für 
die automatisierte Verarbeitung der Tracking-Daten.

