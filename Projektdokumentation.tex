\documentclass[
	ngerman,
	toc=listof, % Abbildungsverzeichnis sowie Tabellenverzeichnis in das Inhaltsverzeichnis aufnehmen
	toc=bibliography, % Literaturverzeichnis in das Inhaltsverzeichnis aufnehmen
	footnotes=multiple, % Trennen von direkt aufeinander folgenden Fußnoten
	parskip=half, % vertikalen Abstand zwischen Absätzen verwenden anstatt horizontale Einrückung von Folgeabsätzen
	numbers=noendperiod % Den letzten Punkt nach einer Nummerierung entfernen (nach DIN 5008)
]{scrartcl}
\pdfminorversion=5 % erlaubt das Einfügen von pdf-Dateien bis Version 1.7, ohne eine Fehlermeldung zu werfen (keine Garantie für fehlerfreies Einbetten!)
\usepackage[utf8]{inputenc} % muss als erstes eingebunden werden, da Meta/Packages ggfs. Sonderzeichen enthalten
\usepackage[T1]{fontenc} % Wichtig für die Verwendung von T1-kodierten Schriftarten
\usepackage{uarial} % Paket laden, das die Schriftart bereitstellt
\renewcommand{\sfdefault}{ua1} % Setze die Standardschriftart auf Arial
\renewcommand{\familydefault}{\sfdefault} % Setze die Standardschriftfamilie auf serifenlos (Arial)


% !TEX root = Projektdokumentation.tex

% Hinweis: der Titel muss zum Inhalt des Projekts passen und den zentralen Inhalt des Projekts deutlich herausstellen
\newcommand{\titel}{Automatisierte Verfolgung und Benachrichtigung für verbesserte Liefertransparenz}
\newcommand{\untertitel}{Überwachungssystem für Warenlieferungen}
\newcommand{\kompletterTitel}{\titel{} -- \untertitel}

\newcommand{\autorName}{Jens Lange}
\newcommand{\autorAnschrift}{Hoßberg 17}
\newcommand{\autorOrt}{21376 Salzhausen}

\newcommand{\betriebLogo}{LogoBetrieb.pdf}
\newcommand{\betriebName}{\textsc{Vonmählen GmbH} }
\newcommand{\betriebAnschrift}{Vor dem Bardowicker Tore 49}
\newcommand{\betriebOrt}{21339 Lüneburg}

\newcommand{\ausbildungsberuf}{Fachinformatiker für Anwendungsentwicklung}
\newcommand{\betreff}{Dokumentation zur betrieblichen Projektarbeit}
\newcommand{\pruefungstermin}{Winter 2024}
\newcommand{\abgabeOrt}{Lüneburg}
\newcommand{\abgabeTermin}{27.11.2024}
 % Metadaten zu diesem Dokument (Autor usw.)
\input{Allgemein/Packages} % verwendete Packages
% Sprachdefinition für JavaScript
\lstdefinelanguage{JavaScript}{
  keywords={typeof, new, true, false, catch, function, return, null, catch,
    switch, var, if, in, while, do, else, case, break},
  keywordstyle=\color{blue}\bfseries,
  ndkeywords={class, export, boolean, throw, implements, import, this},
  ndkeywordstyle=\color{darkgray}\bfseries,
  identifierstyle=\color{black},
  sensitive=false,
  comment=[l]{//},
  morecomment=[s]{/*}{*/},
  commentstyle=\color{purple}\ttfamily,
  stringstyle=\color{red}\ttfamily,
  morestring=[b]',
  morestring=[b]"
}
\lstdefinelanguage{AL}
{
  sensitive=false,
  morekeywords=[1]{
    },
  morekeywords=[2]{
    },
  morekeywords=[3]{
    },
  morekeywords=[4]{
    },
  morekeywords=[5]{
    },
  morecomment=[l]{//},
  morecomment=[s]{/*}{*/},
  morestring=[b]"
}

% Stil für AL-Code
\lstset{
    language=AL,
    basicstyle=\small\ttfamily,
    keywordstyle=[1]\color{blue}\bfseries,
    keywordstyle=[2]\color{purple},
    keywordstyle=[3]\color{green!60!black},
    keywordstyle=[4]\color{green!60!black},
    keywordstyle=[5]\color{orange},
    commentstyle=\color{gray},
    stringstyle=\color{orange},
    breaklines=true,
    numbers=left,
    numberstyle=\small,
    frame=tb,
    columns=fullflexible,
    showstringspaces=false,
    tabsize=4,
    backgroundcolor=\color{gray!10},
    captionpos=b
}
\input{Allgemein/Seitenstil} % Definitionen zum Aussehen der Seiten
\input{Allgemein/Befehle} % eigene allgemeine Befehle, die z.B. die Arbeit mit LaTeX erleichtern
\input{Befehle} % eigene projektspezifische Befehle, z.B. Abkürzungen usw.

\begin{document}
% ---------------------------------------------------------------------------
%\phantomsection
%\thispagestyle{plain}
%\pdfbookmark[1]{Eidesstattliche Erklärung}{ihkdeckblatt}
%\includegraphicsKeepAspectRatio{DeckblattIHK.pdf}{1}
%\cleardoublepage

\phantomsection
\thispagestyle{plain}
\pdfbookmark[1]{Deckblatt}{deckblatt}
\input{Deckblatt}
\cleardoublepage

% Preface --------------------------------------------------------------------
\phantomsection
\pagenumbering{Roman}
\pdfbookmark[1]{Inhaltsverzeichnis}{inhalt}
\tableofcontents

\cleardoublepage

\phantomsection
\listoffigures
\cleardoublepage

\phantomsection
\listoftables
\cleardoublepage

%\phantomsection
%\lstlistoflistings
%\cleardoublepage

\newcommand{\abkvz}{Abkürzungsverzeichnis}
\renewcommand{\nomname}{\abkvz}
\section*{\abkvz}
\markboth{\abkvz}{\abkvz}
\addcontentsline{toc}{section}{\abkvz}
% !TEX root = Projektdokumentation.tex

% Es werden nur die Abkürzungen aufgelistet, die mit \ac definiert und auch benutzt wurden. 
%
% \acro{VERSIS}{Versicherungsinformationssystem\acroextra{ (Bestandsführungssystem)}}
% Ergibt in der Liste: VERSIS Versicherungsinformationssystem (Bestandsführungssystem)
% Im Text aber: \ac{VERSIS} -> Versicherungsinformationssystem (VERSIS)

% Hinweis: allgemein bekannte Abkürzungen wie z.B. bzw. u.a. müssen nicht ins Abkürzungsverzeichnis aufgenommen werden
% Hinweis: allgemein bekannte IT-Begriffe wie Datenbank oder Programmiersprache müssen nicht erläutert werden,
%          aber ggfs. Fachbegriffe aus der Domäne des Prüflings (z.B. Versicherung)

% Die Option (in den eckigen Klammern) enthält das längste Label oder
% einen Platzhalter der die Breite der linken Spalte bestimmt.
\begin{acronym}[WWWWW]
    \acro{AL-Code}{Application Language Code}
    \acro{BC}{Business Central, Software für ERP-Systeme, welches für Warenwirtschaft, Vertrieb, Service und Finanzen genutzt wird}
    \acro{CRM}{Customer Relationship Management}
    \acro{Dynamics}{Microsoft Dynamics CRM ist eine leistungsstarke CRM-Plattform, die Sie dabei unterstützt, den Vertriebsprozess zu automatisieren und zu optimieren}
    \acro{ERP}{Enterprise Resource Planning}
    \acro{NAS}{Network Attached Storage, einfach zu verwaltender Dateiserver}
    \acro{SCM}{Supply Chain Management}
    \acro{API}{Application Programming Interface}   
    \acro{CSV}{Comma Separated Values}
    \acro{JSON}{JavaScript Object Notation}
    \acro{VM}{VONMÄHLEN}
    \acro{SQL}{Structured Query Language}
    \acro{ZF}{Zeitfracht}
    \acro{MVC}{Model-View-Controller}   
    \acro{IDE}{Integrated Development Environment}
    \acro{Dbeaver}{Datenbankverwaltungstool}  
\end{acronym}


%\ac{IDE}
%\Anhang{app:DummyData}
%\ref{app:sdfsf}



\clearpage

% Inhalt ---------------------------------------------------------------------
\pagenumbering{arabic}
\input{Inhalt.tex}

% Literatur ------------------------------------------------------------------
\clearpage
\renewcommand{\refname}{Literaturverzeichnis}
\bibliography{Bibliographie}
\bibliographystyle{Allgemein/natdin} % DIN-Stil des Literaturverzeichnisses


% Anhang ---------------------------------------------------------------------
\clearpage
\appendix
\pagenumbering{roman}
% !TEX root = Projektdokumentation.tex
\section{Anhang}

\subsection{Lastenheft}
\label{app:Lastenheft}
\begin{figure}[htb]
\centering
\includegraphicsKeepAspectRatio{Lastenheft_snipped.PNG}{.8}
\caption{Auszug Lastenheft}
\end{figure}
\clearpage
\clearpage

\subsection{Komplette Zeitplanung}
\label{app:ZeitplanungKomplett}

\begin{tabularx}{\textwidth}{Xrrr}
\label{app:Zeitplanung}\\
\rowcolor{heading}\textbf{Analysephase} & \textbf{} & \textbf{} & \textbf{10 h} \\
1. Erhebung der Anforderungen für die Speicherung und Verarbeitung der Trackingdaten &       &    & 3 h  \\
\rowcolor{odd}2. Analyse der bestehenden Systeme und Identifikation von Schwachstellen &       &    & 4 h  \\
3. Ermittlung der Projektkosten und Berechnung der Einsparungen &       &    & 2 h \\
\rowcolor{odd}4. Erstellung des Wirtschaftlichkeitsberichts &       &    & 1 h  \\
\rowcolor{heading}\textbf{Entwurfsphase} & \textbf{} & \textbf{} & \textbf{13 h} \\
1. Entwurf der Architektur für die Datenbank und Integration mit dem ERP-System (Architekturdiagramm) &       &    & 5 h \\
\rowcolor{odd}2. Design der Datenbankstruktur und der Schnittstellen für die Datenübertragung &       &    & 5 h \\
3. Anbindung an das Monitoring- und Benachrichtigungssystem &       &    & 3 h  \\
\rowcolor{heading}\textbf{Implementierungsphase} & \textbf{} & \textbf{} & \textbf{38 h} \\
1. Implementierung der Schnittstellen zur Abholung und Speicherung der Trackingdaten in der Datenbank &       &    & 10 h  \\
\rowcolor{odd}2. Entwicklung der Integration des ERP-Systems zur Synchronisation und Datenabgleich &       &    & 10 h  \\
3. Programmierung der Prüfalgorithmen für die Trackingdaten und Einrichtung des Monitoring-Systems mit Python &       &    & 8 h  \\
\rowcolor{odd}4. Implementierung des Benachrichtigungssystems für die Fachabteilung &       &    & 4 h \\
5. Integration aller Komponenten und Durchführung erster Funktionstests &       &    & 3 h \\
\rowcolor{odd}6. Debugging und Behebung von initialen Fehlern &       &    & 3 h \\
\rowcolor{heading}\textbf{Testphase} & \textbf{} & \textbf{} & \textbf{8 h} \\
1. Durchführung von Funktionstests zur Sicherstellung der korrekten Datenverarbeitung und -integration &       &    & 4 h \\
\rowcolor{odd}2. Validierung der Datenintegrität und Konsistenzprüfungen &       &    & 2 h \\
3. Testen des Monitoring-Systems und der Benachrichtigungsfunktionalität &       &    & 2 h \\
\rowcolor{heading}\textbf{Einführungsphase} & \textbf{} & \textbf{} & \textbf{3 h} \\
1. Deployment des Systems auf die Produktionsumgebung und Durchführung der finalen Systemkonfiguration &       &    & 2 h \\
\rowcolor{odd}2. Schulung der Fachabteilung zur Nutzung des neuen Systems &       &    & 1 h \\
\rowcolor{heading}\textbf{Dokumentation} & \textbf{} & \textbf{} & \textbf{8 h} \\
1. Erstellung der technischen Dokumentation, einschließlich Systemarchitektur und Datenflussdiagramme &       &    & 8 h \\
\hline
\hline
\rowcolor{heading}\textbf{Gesamt} & \textbf{} & \textbf{} & \textbf{80 h} \\
\end{tabularx}


\subsection{Zeitplanung nach Anpassungen}
\begin{tabularx}{\textwidth}{Xrrr}
\label{app:Zeitnachher} \\
\rowcolor{heading}\textbf{Phase} & \textbf{Geplant} & \textbf{Tatsächlich} & \textbf{Differenz} \\
\textbf{Analysephase} & 10 h   & 12 h  & 2 h \\
\rowcolor{odd}\textbf{Entwurfsphase} & 13 h  & 12 h  & -1 h \\
\textbf{Implementierungsphase} & 38 h  & 38 h  &   \\
\rowcolor{odd}\textbf{Testphase} & 8 h   & 6 h   & -2 h  \\
\textbf{Einführungsphase} & 3 h   & 3 h   &  \\
\rowcolor{odd}\textbf{Dokumentation Erstellen} & 8 h   & 9 h  & 1 h \\
\hline
\hline
\rowcolor{heading}\textbf{Gesamt} & 80 h  & 80 h  &  \\
\end{tabularx}


\subsection{PostgreSQL}
\label{app:PostgreSQL}
\begin{figure}[htb]
\centering
\includegraphicsKeepAspectRatio{sql.png}{1}
\caption{DBeaver - SQL Editor}
\end{figure}


\subsection{Airflow}
\label{app:Airflow}
\begin{figure}[htb]
\centering
\includegraphicsKeepAspectRatio{Airflow.png}{1}
\caption{Airflow: Übersicht der Benutzeroberfläche}
\end{figure}


\subsection{Abnahmephase}
\label{app:Abnahmephase}
\begin{figure}[htb]
\centering
\includegraphicsKeepAspectRatio{dummy_daten_sftp.png}{.8}
\caption{Filezilla: Dummy-Daten auf dem SFTP-Server}
\end{figure}
\clearpage









\end{document}


