% !TEX root = Projektdokumentation.tex

% Es werden nur die Abkürzungen aufgelistet, die mit \ac definiert und auch benutzt wurden. 
%
% \acro{VERSIS}{Versicherungsinformationssystem\acroextra{ (Bestandsführungssystem)}}
% Ergibt in der Liste: VERSIS Versicherungsinformationssystem (Bestandsführungssystem)
% Im Text aber: \ac{VERSIS} -> Versicherungsinformationssystem (VERSIS)

% Hinweis: allgemein bekannte Abkürzungen wie z.B. bzw. u.a. müssen nicht ins Abkürzungsverzeichnis aufgenommen werden
% Hinweis: allgemein bekannte IT-Begriffe wie Datenbank oder Programmiersprache müssen nicht erläutert werden,
%          aber ggfs. Fachbegriffe aus der Domäne des Prüflings (z.B. Versicherung)

% Die Option (in den eckigen Klammern) enthält das längste Label oder
% einen Platzhalter der die Breite der linken Spalte bestimmt.
\begin{acronym}[WWWWW]
    \acro{AL-Code}{Application Language Code}
    \acro{BC}{Business Central, Software für ERP-Systeme, welches für Warenwirtschaft, Vertrieb, Service und Finanzen genutzt wird}
    \acro{CRM}{Customer Relationship Management}
    \acro{Dynamics}{Microsoft Dynamics CRM ist eine leistungsstarke CRM-Plattform, die Sie dabei unterstützt, den Vertriebsprozess zu automatisieren und zu optimieren}
    \acro{ERP}{Enterprise Resource Planning}
    \acro{NAS}{Network Attached Storage, einfach zu verwaltender Dateiserver}
    \acro{SCM}{Supply Chain Management}
    \acro{API}{Application Programming Interface}   
    \acro{CSV}{Comma Separated Values}
    \acro{JSON}{JavaScript Object Notation}
    \acro{VM}{VONMÄHLEN}
    \acro{SQL}{Structured Query Language}
    \acro{ZF}{Zeitfracht}
    \acro{MVC}{Model-View-Controller}   
    \acro{IDE}{Integrated Development Environment}
    \acro{Dbeaver}{Datenbankverwaltungstool}  
\end{acronym}


%\ac{IDE}
%\Anhang{app:DummyData}
%\ref{app:sdfsf}


