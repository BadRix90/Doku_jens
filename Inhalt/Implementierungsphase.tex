% !TEX root = ../Projektdokumentation.tex
\section{Implementierungsphase} 
\label{sec:Implementierungsphase}

\subsection{PostgreSQL-Datenbankmodellierung und -Erstellung}
\label{sec:Datenbankmodellierung}

Die PostgreSQL-Datenbank umfasst eine einzige Tabelle namens \textit{bc\_tracking}. 
Aufgrund der einfachen Projektanforderungen war ein komplexes ER-Modell nicht nötig. 
Die Datenbank speichert Tracking-Informationen von Zeitfracht und BC.

Tabelle~\ref{tab:bcTracking} Die Struktur der Tabelle \textit{bc\_tracking} ist wie folgt:
\tabelle{\textit{bc\_tracking}}{tab:bcTracking}{bcTracking}\\


\subsection{Deployment auf dem Hetzner-Server}
\label{sec:Hetzner-Server}

Die PostgreSQL-Datenbank wurde auf einem Hetzner-Server installiert, um eine zuverlässige und performante 
Umgebung für die Anwendung bereitzustellen. Die Verwaltung der Datenbank erfolgt über DBeaver, das sowohl 
lokale als auch entfernte Datenbankverbindungen unterstützt.

\subsection{Testdaten und Verifizierung des Datenbankzugriffs}
\label{sec:VerifizierungTestdaten}

Um die korrekte Funktion der Datenbank und der Anwendungen zu verifizieren, wurden Testdaten verwendet. 
Diese Testdaten wurden in die Tabelle \textit{bc\_tracking} eingefügt und die Anwendungen darauf ausgeführt, 
um sicherzustellen, dass die Daten korrekt verarbeitet und die gewünschten Aktionen, wie das Senden von E-Mails, 
ausgeführt werden. Screenshots der Anwendung in der Entwicklungsphase mit Dummy-Daten befinden sich im Anhang A3 Abbildung 7

\subsection{Implementierung der Geschäftslogik}
\label{sec:Geschaeftslogik}

Die Geschäftslogik wurde in Python implementiert und besteht aus zwei Hauptskripten: trackings.py und \textit{bc\_db.py}. 
Das Skript trackings.py ist verantwortlich für das Einlesen von CSV-Dateien mit Tracking-Daten von Zeitfracht und das 
Speichern dieser Daten in der PostgreSQL-Datenbank. Das Skript führt folgende Schritte aus:
\begin{itemize}
    \item Einlesen aller CSV-Dateien aus einem Quellverzeichnis.
    \item Validierung der Dateiinhalte, insbesondere der notwendigen Spalten.
    \item Verarbeitung jeder Zeile und Einfügen der Daten in die Tabelle \textit{bc\_tracking}.
    \item Bei erfolgreicher Verarbeitung wird die CSV-Datei aus dem Quellverzeichnis gelöscht.
\end{itemize}


Ein wichtiger Bestandteil ist die Validierung der Daten. Hier wird sichergestellt, dass nur gültige
Statuscodes verarbeitet werden:	

\begin{figure}[htb]
    \centering
    \includegraphicsKeepAspectRatio{validate_row.png}{1}
    \caption{Validierung der Statuscodes}
    \label{fig:validateRow}
\end{figure}

Das Skript \textit{bc\_db.py} kümmert sich um die Verarbeitung der in der Datenbank gespeicherten Trackingdaten und 
interagiert mit der Business Central API. 
Die Hauptaufgaben sind:

\begin{itemize}
    \item Überprüfen, ob bestimmte Tracking-Nummern länger als eine definierte Anzahl von Tagen in einem bestimmten Status verweilen.
    \item Senden von Benachrichtigungs-E-Mails, falls diese Bedingungen erfüllt sind
    \item Senden von Benachrichtigungs-E-Mails, falls Fehler auftreten
    \item Aktualisieren von Datensätzen in Business Central über die API.
\end{itemize}


Beispielsweise wird geprüft, ob Sendungen mit Statuscode 30 länger als 7 Tage nicht zugestellt wurden:

\begin{figure}[htb]
    \centering
    \includegraphicsKeepAspectRatio{sql_view.png}{1}
    \caption{Prüfung von Sendungen im Status 30}
    \label{fig:viewStatus}
\end{figure}


Sowohl \textit{trackings.py} als auch \textit{bc\_db.py} nutzen verschiedene Integrationen, um die Funktionalität zu erweitern. 
Zum Beispiel wird das Decorator-Pattern verwendet, um Integrationen einzubinden:

\begin{figure}[htb]
    \centering
    \includegraphicsKeepAspectRatio{dekorator_pattern.png}{0.8}
    \caption{Verwendung von Integrationen}
    \label{fig:Integrationen}
\end{figure}


Dieses Muster ermöglicht eine modulare und erweiterbare Architektur der Anwendung.



Des Weiteren werden Ausnahmen und Fehler sorgfältig behandelt, um die Stabilität der Anwendung zu gewährleisten. 
Zum Beispiel wird beim Aktualisieren von Daten in BC auf mögliche API-Fehler reagiert:

\begin{figure}[htb]
    \centering
    \includegraphicsKeepAspectRatio{update_data.png}{0.8}
    \caption{Fehlerbehandlung bei API-Aufrufen}
    \label{fig:viewAPI}
\end{figure}

Die Klasse \textit{Zf\_bc\_tracking} in \textit{trackings.py} implementiert die Logik zum Einlesen und Verarbeiten der Trackingdaten 
von Zeitfracht. Zusammenfassend wurde die Geschäftslogik unter Verwendung von Python und verschiedenen Bibliotheken wie SQLAlchemy 
für den Datenbankzugriff, Logging für die Protokollierung und benutzerdefinierten Integrationen für die Anbindung externer Systeme realisiert.


























