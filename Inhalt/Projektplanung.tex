% !TEX root = ../Projektdokumentation.tex
\section{Projektplanung} 

\subsection{Projektphasen}
\label{sec:Projektphasen}
Für diese Umsetzung des Projektes standen dem Auszubildenen 80 Stunden zur Verfügung. 
Welche vor Projektbeginn in verschiedene Phasen aufgeteilt wurden. 

Tabelle~\ref{tab:Zeitplanung} zeigt eine grobe Zeitplanung.
\tabelle{Zeitplanung}{tab:Zeitplanung}{ZeitplanungKurz}\\

Eine detailliertere Zeitplanung ist im \Anhang{app:Zeitplanung}

\clearpage


\subsection{Abweichungen vom Projektantrag}
\label{sec:Abweichung Projektantrag}
Es stellte sich während des Projektes heraus das einige Zeiten nicht eingehalten werden konnten. 
Darauf wird im \Anhang{app:Zeitnachher} weiter eingegangen.


\subsection{Ressourcenplanung}
\label{sec:Ressourcenplanung}
Um die Planung der erforderlichen Ressourcen effizient zu gestalten, wurde gezielt auf bereits vorhandene Infrastruktur zurückgegriffen. Die Auswahl der Hardware und Software erfolgte so, dass keine zusätzlichen Anschaffungen erforderlich waren. Hardware war vollständig vorhanden, da alle Mitarbeiter über einen ausgestatteten Büroarbeitsplatz mit Standard-Peripherie verfügten. Die Softwareauswahl konzentrierte sich auf Open-Source-Tools wie PostgreSQL und Python sowie bereits lizenzierten Programme (Business Central API), die während der Projektlaufzeit genutzt wurden, um die Datenverarbeitung sicherzustellen.
Der Auszubildende übernahm die Hauptverantwortung in der Entwicklung, was insgesamt 80 Stunden veranschlagte. Ein Entwickler wurde für 10 Stunden eingeplant, um gezielt bei technischen Fragen zu unterstützen. Für die infrastrukturelle Vorbereitung, wie das Aufsetzen der Docker-Container inklusive der PostgreSQL-Datenbank und Apache Airflow, wurde Zuarbeit durch die IT-Abteilung geleistet. Dies stellte sicher, dass eine stabile und funktionale Grundlage für die Entwicklung geschaffen wurde.
Die SCM-Abteilung war schließlich dafür verantwortlich, die Anforderungen zu formulieren und das System während der Testphase auf seine Praxistauglichkeit zu überprüfen. Die Kombination aus vorhandenen Ressourcen und gezielter, bedarfsorientierter Planung sorgte dafür, dass die Projektkosten gering blieben und die notwendige Unterstützung jederzeit gewährleistet war.


\subsection{Entwicklungsprozess}
\label{sec:Entwicklungsprozess}
Für die Entwicklung dieses Projekts wurde das Wasserfallmodell als Vorgehensmodell gewählt. 
Das Wasserfallmodell zeichnet sich durch eine klare und lineare Abfolge von Phasen aus, was insbesondere bei Projekten mit fest 
definierten Anforderungen und einem strukturierten Ablauf von Vorteil ist, zudem ist es ein Vorgehensmodell der klassischen Softwareentwicklung, bei dem die Projektphasen sequentiell 
durchlaufen und bearbeitet werden (vgl. \cite{FittkauRuf2008} S. 31). 
Da die Anforderungen an das System im Vorfeld detailliert analysiert und dokumentiert wurden, ermöglichte das Wasserfallmodell 
eine schrittweise und systematische Umsetzung der Projektphasen von der Anforderungsanalyse über die Implementierung bis hin zur abschließenden Testphase.

